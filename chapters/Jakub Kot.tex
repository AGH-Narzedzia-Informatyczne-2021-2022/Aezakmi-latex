\section{Jakub Kot}
\subsection{Problem kanapki}
Wszystko rozpoczęło się od pytania:
\begin{center}
    \textit{Czy hotdog jest kanapką?}
\end{center}
Nowy York stwierdził, że w istocie tak jest.
\begin{figure}[htbp] 
    \begin{center}
    \includegraphics[width=110mm]{pictures/sandwich.png}
    \end{center}
    \caption{Oficjalna odpowiedź Departamentu Stanu}
    \label{fig:ny}
\end{figure}
\newpage
\subsection{Początek wojny}
Gdy wszyscy myśleli, że stosunkowo prosty incydent z kanapkami został rozwiązany, szybko zaczęły wyrastać nowe problemy. Pewien użytkownik znanego serwisu społecznościowego Twitter zapostował opinię:
\begin{center}
    \textit{Ok, skoro problem z kanapkami mamy już z głowy, to proponuję poruszyć nowy temat. Pop Tarts (pic \ref{fig:poptart}) są odmianą ravioli.}
\end{center}
\begin{figure}[h!] 
    \begin{center}
    \includegraphics[width=50mm]{pictures/poptart.jpg}
    \end{center}
    \caption{Pop Tart}
    \label{fig:poptart}
\end{figure}
\FloatBarrier
Szybko pojawiła się odpowiedź z konta firmy samej w sobie:
\begin{center}
    \textit{Wypraszam sobie, to ravioli jest odmianą Pop Tart.}
\end{center}
Tyle wystarczyło, żeby rozpętać wojnę...
\subsection{Tabela kanapek}
Aby zapanować nad powstałym chaosem, stworzona została Tabela Kategoryzacji Kanapek (tab \ref{tab:chart}).
\begin{table}[]
{\footnotesize
 \makebox[\textwidth][c]{\begin{tabular}{|l|c|c|c|}
\hline
\cellcolor[HTML]{000000}{\color[HTML]{343434} }                                                                                                   & \multicolumn{1}{l|}{\textbf{\begin{tabular}[c]{@{}l@{}}Puryzm składników\\ Składniki typowe dla\\ kanapki (mięso, ser,\\ sałata, warzywa, itp.)\end{tabular}}} & \multicolumn{1}{l|}{\textbf{\begin{tabular}[c]{@{}l@{}}Neutralność składników\\ Może zawierać większy\\ zakres słonych składników\end{tabular}}} & \multicolumn{1}{l|}{\textbf{\begin{tabular}[c]{@{}l@{}}Chaotyczność składników\\ Wszystko może być\\ składnikiem\end{tabular}}} \\ \hline
\textbf{\begin{tabular}[c]{@{}l@{}}Puryzm struktury\\ Dwa kawałki pieczywa/pieczonego\\ produktu z dodatkami pomiędzy\end{tabular}}               & \begin{tabular}[c]{@{}c@{}}Chleb z serem i bekonem\\ to kanapka\end{tabular}                                                                                   & \begin{tabular}[c]{@{}c@{}}Chip butty\\ (chleb z frytkami)\\ to kanapka\end{tabular}                                                             & \begin{tabular}[c]{@{}c@{}}Lody pomiędzy goframi\\ to kanapka\end{tabular}                                                      \\ \hline
\textbf{\begin{tabular}[c]{@{}l@{}}Neutralność strukturalna\\ Pojemnik po obu stronach\\ dodatków, ale niekoniecznie\\ dwuczęściowy\end{tabular}} & \begin{tabular}[c]{@{}c@{}}Sub z Subwaya\\ to kanapka\end{tabular}                                                                                             & Hot dog to kanapka                                                                                                                               & Taco z lodami to kanapka                                                                                                        \\ \hline
\textbf{\begin{tabular}[c]{@{}l@{}}Chaotyczna struktura\\ Dowolna żywność\\ otoczona dowolną\\ inną żywnością\end{tabular}}                       & \begin{tabular}[c]{@{}c@{}}Tortilla z kurczakiem\\ to kanapka\end{tabular}                                                                                     & Burrito to kanapka                                                                                                                               & Pop Tart to kanapka                                                                                                             \\ \hline
\end{tabular}
}}\label{tab:chart}
\end{table} 
Niestety jednak to nie wystarczyło, a kłótnie wciąż trwały.
\newpage
\subsection{Prorok}
Spośród tego chaosu wyłoniła się osoba, której udało zakończyć się wojnę. Użytkownik Twittera zmęczony ciągłymi walkami opracował zasadę rozróżniania potraw, nazwaną Regułą Kostki.
\begin{figure}[htbp] 
    \begin{center}
    \includegraphics[width=110mm]{pictures/cuberule.jpg}
    \end{center}
    \label{fig:cuberule}
\end{figure}\\
Dzięki tej prostej zasadzie można rozróżnić rodzaje jedzenia wyłącznie na podstawie lokalizacji skrobii strukturalnej. Oryginalna zasada wyróżnia 6 rodzajów:
\textbf{
\begin{enumerate}
\item Tost
\item Kanapka
\item Taco
\item Sushi
\item Quiche
\item Calzone
\end{enumerate}}
\newpage
\subsection{Przykłady}
{\LARGE Tost}
\textit{Tylko dolna podstawa zawiera skrobię strukturalną}
\begin{itemize}
    \item pizza
    \item nigiri sushi
    \item kawałek tarty (zakrzywiony tost)
\end{itemize}
{\LARGE Kanapka}
\textit{Obie podstawy zawierają skrobię strukturalną}
\begin{itemize}
    \item quesadilla
    \item tost
    \item ciasto biszkoptowe z nadzieniem
    \item lasagna (kanapka wielopoziomowa)
\end {itemize}
{\LARGE Taco}
\textit{Podstawa oraz dwie, nieprzylegające ściany boczne zawierają skrobię strukturalną}
\begin{itemize}
    \item hot dog
    \item kanapka z Subway
\end{itemize}
{\LARGE Sushi}
\textit{Obie podstawy oraz dwie, nieprzylegające ściany boczne zawierają skrobię strukturalną}
\begin{itemize}
    \item enchillada
    \item parówki w cieście
\end{itemize}
{\LARGE Quiche}
\textit{Dolna podstawa oraz wszystkie ściany boczne zawierają skrobię strukturalną}
\begin{itemize}
    \item kebab
    \item zupa w misce zrobionej z bułki
\end{itemize}
{\LARGE Calzone}
\textit{Wszystkie ściany zawierają skrobię strukturalną}
\begin{itemize}
    \item corndog
    \item burrito
    \item pierogi
    \item Pop Tart
\end{itemize}
\newpage
\subsection{Podsumowanie}
Po tym wszystkim pozostaje w głowie już tylko jedno pytanie...\\ \underline{Czy to wszystko ma jakieś znaczenie?}\\
No cóż, obawiam się, że każdy z czytelników musi znaleźć na to pytanie odpowiedź \textit{samemu}. Jeśli mogę coś stwierdzić ze stuprocentową pewnością, to jest to fakt, że \textbf{każde} jedzenie zawsze znajdzie miejsce w moim serduszku.\\
\newcommand{\heart}{\ensuremath\heartsuit}
Jeśli jako J oznaczymy zbiór wszystkich potraw, a \heart   jako moje serce, to zachodzi:
$$\forall_{x\in J}  x\in\heart$$