\section{Robert Barcik}
\subsection{Piasek}
Piasek (rys \ref{fig:Piasek}) to \textbf{skała osadowa}, złożona z niezwiązanych spoiwem ziaren mineralnych, przede wszystkim \underline{kwarcu}. 
Jest najczęściej występującą luźną skałą osadową. Znajduje się \textit{między innymi} w glebie.

Piasek do celów gospodarczych wydobywany jest metodą odkrywkową w kopalniach zwanych piaskowniami oraz przy pogłębianiu rzek, jezior itp. Wykorzystywany jest m.in. do wyrobu szkła.

\begin{figure}[htbp] 
    \centering
    \includegraphics[width=0.9\textwidth]{pictures/Piasek.jpeg}
    \caption{Piasek}
    \label{fig:Piasek}
\end{figure}

\subsection{Rodzaje piasku}
W zależności od rodzaju czynnika transportującego i obrabiającego materiał skalny rozróżnia się piaski m.in.: 
\begin{itemize}
\item rezydualne
\item morskie
\item rzeczne
\item eoliczne
\item fluwioglacjalne
\end{itemize}

\subsection{Przypadkowa tabela}
Patrz na tabelę (tab\ref{tab:sand})
\begin{table}[ht]
\centering
\begin{tabular}{||c c c c||} 
 \hline
 Miejsce & Dokładne miejsce & Ilość & Miły w dotyku \\ [0.5ex] 
 \hline\hline
  Polska & Gleba & Trochę & Polski \\ 
 \hline
 Sahara & Wszędzie & Dużo & Nie \\
 \hline
 Włochy & Plaża & Tyle co na plaży & Czegoś mu brakuje \\
 \hline
\end{tabular}
\label{tab:sand}
\caption{Występowanie piasku}
\end{table}

\subsection{Ciekawostka}
Możesz napisać na piasku wyrażenie matematyczne np.\[ a^2 + b^2 = c^2 \]

\subsection{Top 5 piasków}
\begin{enumerate}
\item Polski
\item Z wody
\item Z polskiej plaży
\item Z piaskownicy
\item Z pustyni
\end{enumerate}