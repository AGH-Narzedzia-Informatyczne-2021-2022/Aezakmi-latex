\section{Miłosz Góralczyk}
\subsection{Sytuacja}
To miasto jest za małe na nas dwóch (see pic \ref{fig:cheems})

\begin{figure}[htbp] 
    \centering
    \includegraphics[width=0.9\textwidth]{pictures/cheems.jpg}
    \caption{cheems}
    \label{fig:cheems}
\end{figure}
\subsection{Przedstawiając za pomocą tabeli}
warto to również przedstawić w tabeli (tab\ref{tab:pops})

\begin{table}[ht]
\centering
\begin{tabular}{||c c c c||} 
 \hline
 Miasto & Ludność & Ja & Ty \\ [0.5ex] 
 \hline\hline
  Londyn & 8 961 989 & 1 & 1 \\ 
 \hline
 Radom & 0 & 0 & 0 \\
 \hline
 Naprawa & 2023 & 1 & 1 \\
 \hline
 To miasto & 1 & 1 & 0 \\
 \hline
\end{tabular}
\label{tab:pops}
\caption{Liczebność wybranych miast}
\end{table} 
\subsection{Zapisując to językiem matematycznym}
$$The City =Me\oplus  You$$
można też to zapisać tak: $(\neg Me\wedge_{}^{} You)\vee ( Me\wedge_{}^{} \neg You)$

\subsection{Mieszczenie miasta}
Pomimo tego, co można wywnioskować z tekstu tym mieście mieści się wiele rzeczy. By wymienić choć kilka można podać:
\begin{enumerate}
\item Bank
\item Saloon
\item Wieża ciśnień
\item Ratusz
\item Radom
\item Ja
\end{enumerate}
Za to z rzeczy które się nie mieszczą w tym mieście, tak jak podane w tabeli wyżej(\ref{tab:pops}), są:
\begin{itemize}
\item Naprawa
\item Londyn
\item Ty
\end{itemize}
\subsection{Podsumowanie}
Mam nadzieję że powyższy dokument pomógł ci \underline{nakreślić} obecną sytuację i wyjaśnić dlaczego Ja (\ref{fig:cheems})  i ty musimy teraz \textbf{walczyć na śmierć i życie} przed ratuszem.